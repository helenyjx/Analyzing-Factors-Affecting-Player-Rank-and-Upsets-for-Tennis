% Options for packages loaded elsewhere
\PassOptionsToPackage{unicode}{hyperref}
\PassOptionsToPackage{hyphens}{url}
%
\documentclass[
]{article}
\usepackage{amsmath,amssymb}
\usepackage{lmodern}
\usepackage{iftex}
\ifPDFTeX
  \usepackage[T1]{fontenc}
  \usepackage[utf8]{inputenc}
  \usepackage{textcomp} % provide euro and other symbols
\else % if luatex or xetex
  \usepackage{unicode-math}
  \defaultfontfeatures{Scale=MatchLowercase}
  \defaultfontfeatures[\rmfamily]{Ligatures=TeX,Scale=1}
\fi
% Use upquote if available, for straight quotes in verbatim environments
\IfFileExists{upquote.sty}{\usepackage{upquote}}{}
\IfFileExists{microtype.sty}{% use microtype if available
  \usepackage[]{microtype}
  \UseMicrotypeSet[protrusion]{basicmath} % disable protrusion for tt fonts
}{}
\makeatletter
\@ifundefined{KOMAClassName}{% if non-KOMA class
  \IfFileExists{parskip.sty}{%
    \usepackage{parskip}
  }{% else
    \setlength{\parindent}{0pt}
    \setlength{\parskip}{6pt plus 2pt minus 1pt}}
}{% if KOMA class
  \KOMAoptions{parskip=half}}
\makeatother
\usepackage{xcolor}
\usepackage[margin=1in]{geometry}
\usepackage{graphicx}
\makeatletter
\def\maxwidth{\ifdim\Gin@nat@width>\linewidth\linewidth\else\Gin@nat@width\fi}
\def\maxheight{\ifdim\Gin@nat@height>\textheight\textheight\else\Gin@nat@height\fi}
\makeatother
% Scale images if necessary, so that they will not overflow the page
% margins by default, and it is still possible to overwrite the defaults
% using explicit options in \includegraphics[width, height, ...]{}
\setkeys{Gin}{width=\maxwidth,height=\maxheight,keepaspectratio}
% Set default figure placement to htbp
\makeatletter
\def\fps@figure{htbp}
\makeatother
\setlength{\emergencystretch}{3em} % prevent overfull lines
\providecommand{\tightlist}{%
  \setlength{\itemsep}{0pt}\setlength{\parskip}{0pt}}
\setcounter{secnumdepth}{-\maxdimen} % remove section numbering
\ifLuaTeX
  \usepackage{selnolig}  % disable illegal ligatures
\fi
\IfFileExists{bookmark.sty}{\usepackage{bookmark}}{\usepackage{hyperref}}
\IfFileExists{xurl.sty}{\usepackage{xurl}}{} % add URL line breaks if available
\urlstyle{same} % disable monospaced font for URLs
\hypersetup{
  pdftitle={Group Project Part I},
  pdfauthor={Gray Team (Nick Carroll, Emmanuel Ruhamyankaka, Jiaxin Ying, Song Young Oh)},
  hidelinks,
  pdfcreator={LaTeX via pandoc}}

\title{Group Project Part I}
\author{Gray Team (Nick Carroll, Emmanuel Ruhamyankaka, Jiaxin Ying,
Song Young Oh)}
\date{2022-10-21}

\begin{document}
\maketitle

\hypertarget{data-overview}{%
\subsubsection{Data Overview}\label{data-overview}}

The dataset is about the men's ATP tour-level main draw matches in the
year of 2021. The data, sourced by Jeff Sackmann of Tennis Abstract,
contains a sample size of 2,733 matches. In the dataset, there are a
total of 49 variables which include both continuous ones (such as the
winner/loser's number of serve points) and categorical ones (such as
surface types of tennis courts). Based on this data, we seek to examine
1) if we can predict a tennis player's rank by the factors such as
height, age, nationality, and handedness and 2) identify which factors
have the most significant impact of increasing the likelihood of an
upset. Here, we have defined an upset as when the winner held a rank of
at least 10 ranks lower than the loser.

The data consists of the basic information for each match. The data can
be grouped into two parts: data that describes the match's attributes,
such as tournament name and surface, and data that describes the results
of the match broken down by the winner's attributes and performance
separately from the loser's attributes and performance. The data
includes a date and a match-specific numeric identifier for the
tournament. Additionally, there is information describing the tournament
level which is broken down into five categories (`G' = Grand Slams, `M'
= Masters 1000s, `A' = other tour-level events, `D' = Davis Cup, `F' =
Tour finals and other season-ending events). Most of the data is a type
`A' tournament (57\%), followed by the `M' and `G' tournaments which
each make up 19\% of the data.

\begin{center}\includegraphics[width=0.8\linewidth]{Group-Project-Part-I_files/figure-latex/unnamed-chunk-1-1} \end{center}

The player-specific data covers a wide range of variables about the
winner and loser, including their name, age, height, nationality, and
handedness (`R' = right, `L' = left, `U' = unknown). The dataset also
includes a player's ATP ranking information at the tournament date. Data
analysis shows that each winner/loser's rank appears to have a heavily
right-skewed distribution due to the concept of rank. However, the
difference between the two player's ranks for each match has a
reasonably normal distribution, as shown in the histogram on the next
page. Its median is estimated as a positive value of 19.

\begin{center}\includegraphics[width=0.8\linewidth]{Group-Project-Part-I_files/figure-latex/unnamed-chunk-2-1} \end{center}

Finally, the portion of the data describing the winner's and losers'
performances include each player's number of first serves made and the
break points saved. The boxplot below shows the winner's number of
first-serve points won in each surface type. The grass type of tennis
court has the highest points based on the median value, and this result
has also been found in the loser's case, which implies that tennis
players are more likely to have more first-serve points when they play
on the grass compared to clay or hard court. A strong positive
relationship exists in the scatterplot on the right side between the
winner's number of serve points and match length in minutes.

\begin{center}\includegraphics[width=0.8\linewidth]{Group-Project-Part-I_files/figure-latex/unnamed-chunk-3-1} \end{center}

\hypertarget{review-of-primary-relationship-of-interest}{%
\subsubsection{Review of Primary Relationship of
Interest}\label{review-of-primary-relationship-of-interest}}

\hypertarget{prediciton-of-a-tennis-players-rank-by-height-age-nationality-and-handedness}{%
\paragraph{Prediciton of a Tennis Player's Rank by Height, Age,
Nationality, and
Handedness}\label{prediciton-of-a-tennis-players-rank-by-height-age-nationality-and-handedness}}

To see if there is a relationship between a player's rank and his
natural characteristics (height, age, nationality, and handedness),
scatter plots are shown below of the player's rank against each of these
characteristics. Since rank is a dependent variable that changes over
time according to a player's performance, these scatter plots show each
player's average rank for the season against his attributes.
Unfortunately, the scatter plots do not show any real correlation
between height or handedness and a player's rank. Age also doesn't
appear to have a strong correlation; however, there is a very loose
correlation with older players having a better rank. This seems to make
sense, as you would not expect older players to continue playing unless
they are still capable of competing at a high level. The plots also
don't show a clear correlation between nationality and rank; however,
while most nationalities do not have a correlation with rank, there are
clearly some nationalities that tend to have higher ranks than others.

\begin{center}\includegraphics[width=0.8\linewidth]{Group-Project-Part-I_files/figure-latex/unnamed-chunk-4-1} \end{center}

\begin{center}\includegraphics[width=0.8\linewidth]{Group-Project-Part-I_files/figure-latex/unnamed-chunk-4-2} \end{center}

\hypertarget{factors-that-can-lead-to-an-upset}{%
\paragraph{Factors that can lead to an
upset}\label{factors-that-can-lead-to-an-upset}}

To review the factors that lead to an upset, bar plots show the
comparative number of upsets versus categorical variables, and line
plots show the comparative number of upsets versus continuous variables.
The bar plots do not establish a clear relationship between any of the
categorical variables (the ``F'' type tournament is a small sample size)
and the likelihood of an upset. The continuous variables also do not
appear to be a clear indicator of upsets. There does appear to be more
upsets in the older and younger players, but this is also potentially
because these age ranges are more likely to compete at the professional
level if they are particularly skilled. There also appears to be a
potential relationship between upsets and height in the extremes, but
extreme heights are also small sample sizes.

\begin{center}\includegraphics[width=0.8\linewidth]{Group-Project-Part-I_files/figure-latex/unnamed-chunk-5-1} \end{center}

\begin{center}\includegraphics[width=0.8\linewidth]{Group-Project-Part-I_files/figure-latex/unnamed-chunk-6-1} \end{center}

\hypertarget{other-characteristics}{%
\subsubsection{Other Characteristics}\label{other-characteristics}}

There are some other variables which are not considered to be used for
answering our research questions (predicting the player's rank and upset
likehood), but they are still worth to be mentioned as the specific
description of their column names. As a result, we divided them into
three different types, which are Tourney information, Winner/Loser
Inforamtion and Match Information.

\emph{Tourney information:}

\begin{itemize}
\tightlist
\item
  tourney\_id: It is a unique identifier for each tournament, such as
  2021-0096. While the first four characters represent the specific
  year, the rest of the number doesn't follow a predictable structure.
\item
  tourney\_name: Tournament name, such as Tokyo Olympics, Munich, and
  Queen's Club.
\end{itemize}

\emph{Winner/Loser Inforamtion:}

\begin{itemize}
\tightlist
\item
  score: It is a palyer's final score.
\item
  winner/loser\_seed: Seed of match winner/loser.
\item
  winner/loser\_entry: There are five main types of entry: `WC' = wild
  card, `Q' = qualifier, `LL' = lucky loser, `PR' = protected ranking,
  `ITF' = ITF entry. Some others are occasionally used.
\item
  winner/loser\_name: The name of winner/loser.
\item
  w/l\_ace: winner's/loser's number of aces.
\item
  w/l\_df: winner's/loser's number of doubles faults.
\item
  w/l\_svpt: winner's/loser's number of serve points.
\item
  w/l\_1stIn: winner's/loser's number of first serves made.
\item
  w/l\_1stWon: winner's/loser's number of first-serve points won.
\item
  w/l\_2ndWon: winner's/loser's number of second-serve points won.
\item
  w/l\_SvGms: winner's/loser's number of serve games.
\item
  w/l\_bpSaved: winner's/loser's number of break points saved.
\item
  w/l\_bpFaced: winner's/loser's number of break points faced.
\end{itemize}

\emph{Match Information:}

\begin{itemize}
\tightlist
\item
  best\_of: A maximum of sets can be played to decide the outcome. There
  are two types: `3' or `5', indicating the number of sets for this
  match. Best of 5 means - whoever is first to win total 3 sets, wins.
  Best of 3 means - whoever is first to win total 2 sets, wins.
\item
  minutes: It is the length of match.
\end{itemize}

\hypertarget{potential-challenges}{%
\subsubsection{Potential Challenges}\label{potential-challenges}}

From the preliminary EDA, there doesn't seem to be a strong correlation
between the predictor and dependent variables. The variables that
correlate with rank tend to have a small sample size of observations in
the data. For example, there appears to be a slight correlation with
senior players having a better rank; however, it is uncertain if this is
an accurate predictor or if there are confounding variables because the
older players tend to continue playing because they are great players.
Rank is particularly difficult to work with because it is ordinal, must
be unique, and is a time-series variable. This analysis assumed that
rank was a dependent variable on a player's attributes and reviewed the
average rank for a player over the entirety of the season; however, a
player's average rank may not be an accurate representation of his
actual rank because his rank may have changed for a portion of the
season due to injuries. Furthermore, while rank should be ordinal and
unique, average rank is not required to be either ordinal or unique.

Many of the variables of interest have some missing values, which will
cause noise during the modeling stage. For example, handedness has
several missing values, meaning several players are considered neither
left-handed nor right-handed in the model. These players will likely be
dropped during the modeling phase.

\end{document}
